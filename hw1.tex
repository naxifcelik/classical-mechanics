\documentclass[12pt]{article}

\title{ \texttt{klasik mekanik ödevi 1}}
\author{Nazif ÇELİK 090200712}
\date{07.03.2023}
\begin{document}
\maketitle
\newpage
\section{A}
a) Parçacık sürtünme olmadan kaydığından, ona etki eden tek kuvvet korunumlu bir kuvvet olan yerçekimidir. Bu nedenle, parçacığın t=0'daki ilk mekanik enerjisi E, kinetik ve potansiyel enerjisinin toplamı ile verilir:
\bigbreak
E = 1/2 mv0^2 - mgR cos(\phi_0)

Burada m parçacığın kütlesi ve v0 başlangıç hızının büyüklüğüdür.
\bigbreak
Parçacığın hareketi dairesel bir yola sınırlıdır, bu nedenle herhangi bir t zamanındaki ivmesi merkezcil ivme ile verilir:
\bigbreak
 
 a_c = v^2/R            
 
 burada v, t zamanındaki hızının büyüklüğüdür.


\bigbreak
Parçacığın ilk ivmesini belirlemek için, $\beta$ ve $\phi$ cinsinden başlangıç hızını $v_0$ bulmamız gerekir. 
$v_0$'ı $\phi$ açısıyla ilişkilendirmek için enerjinin korunumunu kullanabiliriz:
\bigbreak
E = 1/2 mv0^2 - mgR cos(\phi) = sabit

burada $\phi_0$'ı $\phi$ ile değiştirdik, çünkü enerji korunumu denklemi herhangi bir t değeri için geçerlidir.
\bigbreak

$v_0$ için çözerek şunu elde ederiz:
\bigbreak
v_0^2 = 2gR(cos(\phi) - cos(\phi_0))


Bu ifadeyi merkezcil ivme denkleminde yerine koyarsak şunu elde ederiz:
\bigbreak
a_c = 2g(cos(\phi) - cos(\phi_0))
\bigbreak
Son olarak, $cos(\phi)$'yi trigonometri kullanarak $\beta$ cinsinden ifade edebiliriz:
\bigbreak
cos(\phi) = cos(\phi)cos(\phi_0) + sin(\beta)sin(\phi_0)

Bu ifadeyi önceki denklemde yerine koyarak, parçacığın ilk ivmesini $\beta$ ve $\phi$ cinsinden elde ederiz:


a_0 = 2g(cos(\beta)cos(\phi_0) + sin(\beta)sin(\phi_0) - cos(\phi_0)) = 2g(cos(\beta) - 1)cos(\phi_0) + 2g sin(\beta)sin(\phi_0 )
\bigbreak
Bu nedenle parçacığın ilk ivmesi, x ekseni ile ilk hız vektörü arasındaki hem $\phi_0$ açısına hem de $\beta$ açısına bağlıdır.
\newpage
\section{B}
Tel, sabit $\Omega$ açısal hızıyla döndüğünden, parçacık, dönme ekseninden radyal olarak dışa doğru etki eden bir merkezkaç kuvvetine maruz kalır. Bu kuvvetin büyüklüğü $mR\Omega^2$'dir ve tel boyunca her noktada tele dik olarak yönlendirilir.
\bigbreak
t=0 anında, parçacığın x yönünde başlangıç hızı v$_0$'dır ve ayrıca negatif y yönünde mg yerçekimi kuvvetine tabidir. Tel döndükçe, parçacık tel boyunca hareket edecek ve hareketini etkileyen değişen bir merkezcil kuvvetle sonuçlanan yerçekimi kuvvetinin değişen yönünü deneyimleyecektir.
\bigbreak
Açısal hız $\Omega$'e ve $\phi$ arasındaki ilişkiyi bulmak için parçacığın tel boyunca hareketini düşünebiliriz. Tel boyunca etki eden yerçekimi kuvvetinin bileşeni şu şekilde verilir:
\bigbreak
F_g = -mg sin(\phi)
\bigbreak
Tel boyunca etki eden merkezkaç kuvvetinin bileşeni şu şekilde verilir:
\bigbreak
F_c = mR\Omega^2 sin(\theta)
\bigbreak
burada $\theta$, tel ile dikey y ekseni arasındaki açıdır. t=0 anında, 
\bigbreak
$\theta = \pi/2 - \phi_0$.
\bigbreak
Tel boyunca etki eden net kuvvet bu nedenle:
\bigbreak
F_{net} = -mg sin(\phi) + mR\Omega^2 sin(\theta)
\bigbreak
Newton'un ikinci yasasını kullanarak şunu yazabiliriz:
\bigbreak
m\ddot\phi = F_{net}
\bigbreak
burada $\ddot\phi$ parçacığın tel boyunca açısal ivmesidir.
\bigbreak
$F_{net}$, $sin(\theta)$ ve $\theta$ ifadelerini yukarıdaki denklemde yerine koyarak şunu elde ederiz:
\bigbreak
m\ddot\phi = -mg sin(\phi) + mR\Omega^2 cos(\phi_0) sin(\phi)
\bigbreak
Bu, $\ddot\phi$ açısal ivmesini $\Omega$ açısal hızı ve $\phi_0$ başlangıç açısı ile ilişkilendiren doğrusal olmayan bir diferansiyel denklemdir.
Parçacığın tel üzerinde maksimum yüksekliğe ulaştığı $\Omega$ kritik değerini $\Omega_c$ bulmak için, $\ddot\phi$ = 0 ayarlayabilir ve $\Omega$ için çözebiliriz:
\bigbreak
\Omega_c^2 = g/(R cos(\phi_0))
\bigbreak
Bu kritik $\Omega$ değeri, mümkün olan minimum $\phi$ değerine karşılık gelir, çünkü bu, parçacığın tel üzerinde maksimum yükseklikte kalacağı $\Omega$ değeridir. $\Omega$ > $\Omega_c$ için, parçacık tel üzerindeki maksimum yükseklik etrafında salınır ve maksimum salınım açısı şu şekilde verilir:
\bigbreak
\phi_{max} = cos^-1(\omeaga_c/\Omega) - \phi_0
\bigbreak
$\Omega < \oemga_c$ için, parçacık telden aşağı kayacak ve açısal ivmesi $\ddot\phi$ negatif olacaktır.
Bu nedenle, $\Omega$ açısal hızı ve $\phi$ açısı arasındaki ilişki, başlangıç açısı $\phi_0$'a bağlı olan kritik açısal hızın $\Omega_c$ değerine bağlıdır. $\Omega > \Omega_c$ ise, parçacık tel üzerindeki maksimum yükseklik etrafında salınacak, $\Omega < \Omega_c$ için ise parçacık telden aşağı kayacaktır.
\newpage
\section{C ve D}
Parçacığın mutlak ivmesini bulmak için, parçacığa radyal ve teğet yönlerde etki eden kuvvetleri çözmemiz gerekir. Parçacığın tel boyunca hareketini tanımlamak için kutupsal koordinatları $(\rho, \phi)$ ve birim vektörleri $(\hat\rho, \hat\phi)$ kullanabiliriz.
\bigbreak
Parçacığın radyal ivmesi şu şekilde verilir:
\bigbreak
a_\rho = -g sin(\phi) + R\Omega^2 cos(\phi)
\bigbreak
Bu, yerçekimi kuvveti ile parçacığa radyal yönde etki eden merkezkaç kuvvetinin toplamıdır.
Parçacığın teğetsel ivmesi şu şekilde verilir:
\bigbreak
a_\phi = R\ddot\phi
Bu, parçacığın tel boyunca açısal ivmesidir ve bu, $\phi$ açısının zamana göre ikinci türevi çarpı R yarıçapına eşittir.
\bigbreak
Bu ivmeleri $\hat\rho$ ve $\hat\phi$ temel vektörleri cinsinden ifade etmek için şu ilişkiyi kullanabiliriz:
\bigbreak
\hat\rho = cos(\phi) \hat{x} + sin(\phi) \hat{y}

\hat\phi = -sin(\phi) \hat{x} + cos(\phi) \hat{y}
\bigbreak

burada $\hat{x}$ ve $\hat{y}$, x ve y yönlerindeki birim vektörlerdir.
\bigbreak
Parçacığın mutlak ivmesi şu şekilde yazılabilir:
\bigbreak
a = a_{\rho} \hat\rho + a_{\phi} \hat{\phi}
\bigbreak
$a_{\rho}$ ve $a_{\phi}$ ifadelerini değiştirerek şunu elde ederiz:
\bigbreak
a = (-g sin(\phi) + R\Omega^2 cos(\phi)) cos(\phi) \hat{\rho} + R\ddot\phi (-sin(\phi)) \hat{\phi}
\bigbreak
Bu ifadeyi basitleştirerek şunu elde ederiz:
\bigbreak
a = (-g cos(\phi) sin(\phi) + R\Omega^2 cos^2(\phi)) \hat{\rho} - R\ddot\phi sin(\phi) \hat{\phi}
\bigbreak
Son olarak, bu ivmeyi verilen \beta×\phi=k temeli cinsinden ifade edebiliriz:
\bigbreak
a = (-g cos(\phi) sin(\phi) + R\Omega^2 cos^2(\phi)) \hat{\beta} - (R\ddot\phi sin(\phi)) \hat{\phi}
\newpage
Burada $\hat{\beta}$, hem $\hat{\rho}$ hem de $\hat{\phi}$'ye dik birim vektördür ve şu şekilde verilir:
\bigbreak
\hat{\beta} = -sin(\phi) \hat{x} + cos(\phi) \hat{y}
\bigbreak
Bu nedenle, parçacığın tel boyunca $\Omega$, R, $\dot\phi$ ve $\ddot\phi$ cinsinden mutlak ivmesi şu şekilde verilir:
\bigbreak
a = (-g cos(\phi) sin(\phi) + R\Omega^2 cos^2(\phi)) \beta×-\phi - (R\ddot\phi sin(\phi)) \phi
\bigbreak
burada $\beta\times\phi$, $\hat{\beta}$ ve $\hat{\phi}$'nin cross productıdır ve k, hareket düzlemine dik birim vektördür.
\newpage
\section{E}
$\phi = \phi/2$ olduğunda anlık ivmeyi bulmak için (d) kısmında türetilen ifadeyi kullanabiliriz:
\bigbreak
a = (-g cos(\phi) sin(\phi) + R\Omega^2 cos^2(\phi)) \beta\times\phi - (R\ddot\phi sin(\phi)) \phi

$\phi = \phi/2$'de şuna sahibiz:
\bigbreak
cos(\phi) = cos(\phi/2)

sin(\phi) = sin(\phi/2)

cos^2(\phi) = cos^2(\phi/2)

sin(\phi) = 2 sin(\phi/2) cos(\phi/2) 

(sin için çift açı formülünü kullanarak)
\bigbreak
Bu değerleri değiştirerek şunu elde ederiz:
\bigbreak
a = (-g/2) sin(\phi/2) cos(\phi/2) \beta\times\phi - (R/2) \ddot\phi sin(\phi/2) \phi
\bigbreak
Bu ifadeyi basitleştirerek şunu elde ederiz:
\bigbreak
a = (-g/4) sin(\phi) \beta\times\phi - (R/2) \ddot\phi sin(\phi/2) \phi
\bigbreak
Bu nedenle, parçacığın $\phi = \phi/2$'deki anlık ivmesi şu şekilde verilir:
\bigbreak
a = (-g/4) sin(\phi/2) \beta\times\phi - (R/2) \ddot\phi sin(\phi/4) \phi
\bigbreak
İvmenin yönünün hareket düzlemine dik olduğuna ve vekörel çarpım $\beta\times\phi$ ile verildiğine dikkat edin. İvmenin büyüklüğü g, R, $\phi/2$, $\ddot\phi$ değerlerine ve $\phi$ değeri tarafından belirlenen $\hat\beta$ ile $\hat\phi$ arasındaki açıya bağlıdır.
\newpage
\section{F}
$\phi = \phi/2$ olduğunda elektrik motorunun ihtiyaç duyduğu torku bulmak için (c) kısmında türetilen açısal ivme ifadesini kullanmamız gerekir:
\bigbreak
\ddot\phi = (-g/R) sin(\phi) + \Omega^2 cos(\phi)

$\phi = \phi/2$'de şuna sahibiz:
\bigbreak
sin(\phi) = sin(\phi/2)

cos(\phi) = cos(\phi/2)

Bu değerleri değiştirerek şunu elde ederiz:
\bigbreak
\ddot\phi = (-g/R) sin(\phi/2) + \Omega^2 cos(\phi/2)
\bigbreak
Elektrik motorunun ihtiyaç duyduğu tork şu şekilde verilir:
\bigbreak
\tau = I\alpha

burada I, telin eylemsizlik momenti ve $\alpha$ açısal ivmedir.

R yarıçaplı ince dairesel bir telin eylemsizlik momenti şu şekilde verilir:
\bigbreak
I = (1/2)MR^2

burada M, telin kütlesidir.

$\ddot\phi$ ifadesini ve eylemsizlik momentini tork denkleminde değiştirerek şunu elde ederiz:
\bigbreak
\tau = (1/2)MR^2 [(-g/R) sin(\phi/2) + \Omega^2 cos(\phi/2)]

Bu ifadeyi basitleştirerek şunu elde ederiz:
\bigbreak
\tau = (1/2)MgR sin(\phi/2) - (1/2)M\Omega^2 R^2 cos(\phi/2)
\bigbreak
Bu nedenle, $\phi = \phi/2$ olduğunda elektrik motorunun ihtiyaç duyduğu tork şu şekilde verilir:
\bigbreak
\tau = (1/2)MgR sin(\phi/2) - (1/2)M\Omega^2 R^2 cos(\phi/2)
\newpage
Elektrik motorunun o anda tükettiği gücü bulmak için şu ifadeyi kullanabiliriz:
\bigbreak
P = \tau\Omega

burada $\Omega$, $\Omega$'a eşit olan telin açısal hızıdır.

Tork için ifadeyi ve açısal hızın değerini güç için denklemde değiştirerek şunu elde ederiz:
\bigbreak
P = [(1/2)MgR sin(\phi/2) - (1/2)M\Omega^2 R^2 cos(\phi/2))]\Omega

Bu ifadeyi basitleştirerek şunu elde ederiz:
\bigbreak
P = (1/2)M\Omega[gR sin(\phi/2) - \OmegaR^2 cos(\phi/2)]

Bu nedenle, $\phi = \phi/2$ olduğunda elektrik motorunun tükettiği güç şu şekilde verilir:
\bigbreak
P = (1/2)M\Omega[gR sin(\phi/2) - \OmegaR^2 cos(\phi/2)]
\newpage
\section{G}
Parçacık parabolik tel boyunca sabit hızla kayarken, teğetsel ivmesi sıfır olmalıdır. Teğetsel ivme şu şekilde verilir:
\bigbreak
a_t = r \alpha

r, parçacığın dönme ekseninden uzaklığı ve $\alpha$, telin açısal ivmesidir.
Parabolik tel için herhangi bir noktadaki eğrilik yarıçapı şu şekilde verilir:
\bigbreak
r = (1 + \dot{y}^2)^(3/2) / |\ddot{y}|

burada $\dot{y}$ ve $\ddot{y}$, y'nin x'e göre birinci ve ikinci türevleridir.

Parabolik tel için elimizde:
\bigbreak
y = x^2/L

\dot{y} = 2x/L

\ddot{y} = 2/L

Bu değerleri değiştirerek şunu elde ederiz:
\bigbreak
r = (1 + (4x^2/L^2))^(3/2) / (4/L)
\newpage
Teğetsel ivme, aşağıdaki formül kullanılarak açısal ivme cinsinden ifade edilebilir:
\bigbreak
\alpha = a_t / r

$a_t = 0$ olduğundan \alpha = 0
\bigbreak
Bu, parçacığın tel boyunca sabit bir hızla kayması için telin açısal hızının sabit olması gerektiği anlamına gelir. Bu nedenle: $\Omega = sabit$
\bigbreak
Dönen bir kovadaki suyun serbest yüzeyinin neden parabolik bir şekil aldığını anlamak için su moleküllerine etki eden kuvvetleri düşünebiliriz. Yerçekimi kuvveti suyu aşağı çekerken, merkezkaç kuvveti suyu dışarı doğru iter. Sonuç olarak, kovanın tabanına yakın su molekülleri kovanın dışına doğru net bir kuvvet uygularken, kovanın tepesine yakın su molekülleri kovanın merkezine doğru net bir kuvvet uygular. Bu kuvvetleri dengelemek için su yüzeyinin şeklinin ayarlanması gerekir. Parabolik bir şekil bu dengeyi sağlar, çünkü yüzeyin eğriliği, merkezkaç kuvvetinin daha güçlü olduğu kovanın dışına yakın yerlerde daha dik ve yerçekimi kuvvetinin daha güçlü olduğu kovanın merkezine yakın yerlerde daha düzdür. Bu nedenle, suyun serbest yüzeyi, sabit hızla dönen kovanın içinde parabolik bir şekil alır.
\end{document}

